\documentclass[10pt,a4paper]{exam} % turn it into a class! 
\usepackage[latin1]{inputenc}
\usepackage{amsmath}
\usepackage{amsfonts}
\usepackage{amssymb}
\usepackage{graphicx}
\usepackage{titlesec}
\usepackage{hyperref}
\usepackage{fancyeq}
\usepackage{tikz}
%\usepackage{tikz-uml}
\usepackage{mathpartir}
\usetikzlibrary{matrix,decorations.pathmorphing,shapes,arrows,backgrounds,positioning}
\usepackage{graphicx,xcolor}
\usepackage{geometry}
\usepackage{everysel}
\usepackage[normalem]{ulem}
\usepackage{mdframed}
\usepackage{drawstack}

\tikzset{
  treenode/.style = {align=center, inner sep=0pt, text centered,
    font=\sffamily},
  arn_n/.style = {treenode, circle, black, font=\sffamily\bfseries, draw=black,
    fill=white, text width=1.5em},% arbre rouge noir, noeud noir
  arn_r/.style = {treenode, circle, red, draw=red, 
    text width=1.5em, very thick},% arbre rouge noir, noeud rouge
  arn_x/.style = {treenode, rectangle, draw=black,
    minimum width=0.5em, minimum height=0.5em}% arbre rouge noir, nil
}

\usepackage[sc]{mathpazo}
\linespread{1.05}         % Palatino needs more leading (space between lines)
\usepackage[T1]{fontenc}

% some format settings
% for hard-bound final submission, use:
%\setlength{\oddsidemargin}{4.6mm}     % 30 mm left margin - 1 in
% for soft-bound version and techreport, use instead:

\setlength{\oddsidemargin}{-0.4mm}    % 25 mm left margin - 1 in
\setlength{\evensidemargin}{\oddsidemargin}
\setlength{\topmargin}{-5.4mm}        % 20 mm top margin - 1 in
\setlength{\textwidth}{160mm}         % 20/25 mm right margin
\setlength{\textheight}{237mm}        % 20 mm bottom margin
\setlength{\headheight}{5mm}
\setlength{\headsep}{5mm}
\setlength{\parindent}{0mm}
\setlength{\parskip}{\medskipamount}
\renewcommand\baselinestretch{1.2} % thesis format (not needed for techreport)
% don't let large figures hijack entire pages
\renewcommand\topfraction{.9}
\renewcommand\textfraction{.1}
\renewcommand\floatpagefraction{.8}

\pagestyle{headandfoot}
%\pointsinrightmargin
%\pointname{ marks}
%\marginpointname{ marks}

\marksnotpoints 

\definecolor{campurple}{HTML}{862D91} 
\definecolor{campurpledark}{HTML}{2A185C}

\hypersetup{  
  urlcolor=campurple,
  linkcolor=campurple,
  colorlinks=true  
}

\titlelabel{\llap{\thetitle\quad}}

\newcommand {\lbrac} {\makebox[0pt]{[\kern-1ex[}}
\newcommand {\rbrac} {\makebox[0pt]{]\kern-1ex]}}
\newcommand{\denote}[1]{\lbrac~#1~\rbrac}


\def\mystrut(#1,#2){\vrule height #1pt depth #2pt width 0pt} 

\titlespacing*{\section}{0pt}{0pt}{0pt}


\begin{document}

\newcommand{\course}{Compiler\\[-0.4cm]Construction}
\newcommand{\week}{IV}

\everymath{\color{campurpledark}}
\everydisplay{\color{campurpledark}}

%\vspace{15pt}

%\begin{center}
%\emph{Complete SECTION 1 and ONE other section.}
%\end{center}

%\begin{center}
%\emph{Answer SECTION 1 and TWO other sections.}
%\end{center}

\marksnotpoints
\pointsdroppedatright
\marksnotpoints
\marginpointname{ \points\ worth of words}

\begin{center}
\LARGE {\textbf{\color{campurpledark} \course} }\\[-0.2cm]
\Large \color{campurpledark} Supervision \week: Return of the Memory\\
\end{center}

{\color{campurple}\hrule}

\newcommand{\metavar}[1]{{\color{campurple}#1}}

\vspace{0.5cm}

\newcommand{\terminal}[1]{\texttt{\color{campurple}#1}}
\newcommand{\bl}[1]{{\color{black}#1}}

%\section*{Introduction}

\begin{center}
\parbox[c]{340pt}{
A long time ago in a supervision far, far away... \\

We have constructed a compiler which, given an input program in the STG language, produces C code that corresponds to the operational semantics of the source language.\\

Since then, the evil $\mathbf{let}(\mathbf{rec})$ expressions have greedily been allocating memory on the heap but they never release any when it is no longer needed.\\

In these desperate times of memory shortage, it falls on you to reclaim some of the memory....
}
\end{center}

\section*{Exercises}

\begin{questions}
\question Discuss whether a ``type-aware'' dynamic linker might combat ``DLL hell''.
\question With the operational semantics / code generator for the STG machine in mind, suggest \emph{at least one} optimisation our compiler could perform. If you are feeling particularly adventurous, try to implement your optimisation(s).
\question Discuss to what extent (if any) you think that closures are the same as objects. 
\question An eager student from another university has recently learnt about garbage collectors, but he is annoyed by having to interrupt his program's execution to run garbage collection. He plans to implement a garbage collector which runs continuously on a separate thread. Discuss what issues he might face.
\question Describe how you would go about constructing a compiler for the STG language in the STG language itself.
\question A compiler is essentially a translator from one language to another. We say that a compiler is correct if the programs it generates in the target language preserve the meaning of the source programs. Consider the following data type for the abstract syntax of a simple expression language (aka ``Hutton's Razor''):
\begin{displaymath}
\begin{array}{lcl}
\mathbf{data}~\mathit{Expr} & = & \mathit{Val}~\mathit{Int} \mid \mathit{Plus}~\mathit{Expr}~\mathit{Expr}
\end{array}
\end{displaymath}
We give this language a denotational semantics\footnote{There's an entire Part II course on this, woo!} -- that is, one which is defined by mapping terms to objects in some domain -- in the form of a Haskell function:
\begin{displaymath}
\begin{array}{lcl}
\mathit{eval} & :: & \mathit{Expr} \to \mathit{Int} \\
\mathit{eval}~(\mathit{Val}~n) & = & n \\
\mathit{eval}~(\mathit{Plus}~e_0~e_1) & = & \mathit{eval}~e_0 + \mathit{eval}~e_1
\end{array}
\end{displaymath}
We now want to compile this language to a stack-based machine with the following instruction set:
\begin{displaymath}
\begin{array}{lcl}
\mathbf{data}~\mathit{Instr} & = & \mathit{ADD} \mid \mathit{PUSH}~\mathit{Int}
\end{array}
\end{displaymath}
Programs for this machine are lists of instructions:
\begin{displaymath}
\begin{array}{lcl}
\mathbf{type}~\mathit{Program} & = & \hslist{\mathit{Instr}}
\end{array}
\end{displaymath}
Stacks are represented as lists of integers. The semantics of the machine are then as follows:
\begin{displaymath}
\begin{array}{llcl}
\mathit{exec} && :: & \mathit{Program} \to \hslist{Int} \to \mathit{Maybe}~\hslist{Int} \\
\mathit{exec}~\hslist{} & s & = & \mathit{Just}~s \\
\mathit{exec}~(\mathit{ADD} : p) & (y:x:s) & = &\mathit{exec}~p~((x+y):s) \\
\mathit{exec}~(\mathit{ADD} : p) & s & = & \mathit{Nothing} \\
\mathit{exec}~(\mathit{PUSH}~n:p) & s & = & \mathit{exec}~p~(n:s)
\end{array}
\end{displaymath}
Finally, we define a compilation function which turns $\mathit{Expr}$ values into $\mathit{Program}$ values:
\begin{displaymath}
\begin{array}{lcl}
\mathit{comp} & :: & \mathit{Expr} \to \mathit{Program} \\
\mathit{comp}~(\mathit{Val}~n) & = & \hslist{\mathit{PUSH}~n} \\
\mathit{comp}~(\mathit{Plus}~e_0~e_1) & = & \mathit{comp}~e_0 \append \mathit{comp}~e_1 \append \hslist{\mathit{ADD}}
\end{array}
\end{displaymath}
Prove that the correctness property holds for this compiler by structural induction on $e$ (optionally, by using a proof assistant of your choice):
\begin{displaymath}
\begin{array}{llcl}
\forall e :: \mathit{Expr}.\forall s :: \hslist{Int}. \qquad & \mathit{Just}~\hslist{\mathit{eval}~e} & = & \mathit{exec}~(\mathit{comp}~e)
\end{array}
\end{displaymath}


\end{questions}

\section*{Practical Exercises}

For the practical exercises, we will implement a copying garbage collector for the STG language.

\begin{mdframed}
\textbf{Skeleton Code}\\
The skeleton code has been updated for this exercise. It is available as the \texttt{supervision4} branch on the \texttt{mbg/compconstr-code} repository. Note that the \texttt{supervision4} branch \textbf{does not} include answers to the previous exercises.

If you have forked the repository to your own GitHub account, you should first bring your fork up-to-date:
\begin{verbatim}
$ git remote add upstream https://github.com/mbg/compconstr-code
$ git fetch upstream
\end{verbatim}
At this point, you should ensure that you are on your own \texttt{master} branch:
\begin{verbatim}
$ git checkout master
\end{verbatim}
Next, merge the (potentially) updated \texttt{upstream/supervision3} branch into your own:
\begin{verbatim}
$ git merge upstream/supervision3
\end{verbatim}
In theory, you should not encounter any merge errors at this point. If you do, you will have to fix and commit them before proceeding.

Next, you should merge the \texttt{upstream/supervision4} branch into your own:
\begin{verbatim}
$ git pull
$ git merge supervision4
\end{verbatim}
Again, you should not encounter any merge errors, but if there are any, fix and commit them before proceeding.

There are no new dependencies, but the package configuration has changed, so just run the following command to reconfigure it:
\begin{verbatim}
$ cabal configure
\end{verbatim}
Verify that the program compiles successfully (repeat this step whenever you wish to re-compile the compiler):
\begin{verbatim}
$ cabal build
\end{verbatim}
Some functions in the $\mathit{AbstractC}$ module, such as $\mathit{allocMemory}$, are now parametrised by the type of value they work on. You should be largely unaffected by this, but there may be cases where you have to pass an extra argument to them. It should be OK to use any $\mathit{PolyType}$ values which may be floating around in the scope. For example, in the $\mathit{compExpr}$ case for constructor application, just use $t$ as an extra argument to $\mathit{allocMemory}$. If there is nothing in scope, just make one up such as we did before with $\mathit{MonoTy}~\$~\mathit{AlgTy}~\texttt{"\_Cont"}$ to ``type'' continuations in the code generator.

Once the program compiles successfully, this should create an executable in the \texttt{./dist/build/stg/} directory. A new test program has been added to the \texttt{tests/gc/} folder. You can compile it in the usual way:
\begin{verbatim}
$ cd /tests/gc/
$ make
\end{verbatim} 
This test program should happily compile, but it will soon blow up if run (it should cause a crash / infinite loop / who knows).  
\end{mdframed}

\begin{questions}
\question In \texttt{src/CodeGenAnalysis.hs}, complete the definitions of $\mathit{heapCost}$ which, given nodes in the abstract syntax tree, should calculate how much heap space in words will be allocated by the code that is generated for the node. The resulting number of words, $n$, will be used by the heap overflow check in a closure's standard entry code to verify that adding $n$ to the heap pointer, \texttt{Hp}, will not result in a heap overflow. There are a few things you may want to consider when you implement this function:
\begin{itemize}
\item $\mathbf{let}(\mathbf{rec})$ expressions and constructor applications may allocate memory on the heap (see respective cases in the $\mathit{compExpr}$ function in \texttt{src/CodeGen.hs}). Algebraic case alternatives may deallocate memory (see $\mathit{compAlgAltCont}$), but default alternatives for an algebraic case expression leak memory in this implementation (see $\mathit{compAlgDefault}$). 
\item The heap overflow check should only check that there is enough space on the heap for the allocations which the code for the current closure is going to make. You should not account for the space needed by other closures which may be entered in the process of evaluating the current one.
\item If the body of a $\lambda$-form contains a $\mathbf{case}$ expression, then only one of its alternatives will be evaluated, but you cannot predict which. 
\end{itemize}
\question Before we can implement the garbage collector, there is a tiny, little problem: the free variables of dynamic closures on the STG heap may either be pointers to other closures or primitive integers. In other words, when the garbage collector discovers a closure, it will not be able to decide whether it should treat a free variable as a pointer to another closure that needs to be explored or a primitive value which can be ignored.

To circumvent this problem, we finally put the info tables we generated for the previous exercise to use. Currently, info tables are generated for every closure and only contain a pointer to the closure's standard entry code. We will now extend them with two additional pointers: one to a closure's \emph{evacuation} code and one to a closure's \emph{scavenging} code. These functions are generated for each closure, so that the generated code will be aware of which free variables are pointers and which are not.
\begin{parts}

\part The evacuation code of a closure is responsible for copying the closure from the current heap (the ``from-space'') to the other heap (the ``to-space''), then overwriting the closure in the from-space with a \emph{forwarding pointer}, and finally returning the memory address of the closure in the to-space. The forwarding pointer is used to mark a closure as ``copied''. If the garbage collector encounters it again, then the evacuation code for the forwarding pointer will simply return the address of the closure in to-space, instead of copying the closure again.

The corresponding function for this in the code generator is $\mathit{compEvac}$ (at the bottom of \texttt{src/CodeGen.hs}). Complete the implementation of this function. Some hints:
\begin{itemize}
\item When the evacuation code is called, the \texttt{Node} register will point to the closure in from-space that should be copied to to-space. The heap pointer register, \texttt{Hp}, will point to the last used location in to-space.
\item Functions you have encountered in the previous exercise, such as $\mathit{allocMemory}$ and $\mathit{writeHeap}$, can still be used here. Since both generate code which makes use of \texttt{Hp}, both operate on the to-space.
\item There is no direct way to access the from-space, but since \texttt{Node} still points to a location in from-space, you can index into it. Functions, such as $\mathit{writeHeap}$, which accept a value of type $\mathit{Symbol}~a$ as argument allow you to do this. For example, if you wanted to write the current closure's info pointer to the address pointed at by \texttt{Hp}, you could write:
\begin{verbatim}
writeHeap 0 (IndexSym (RegisterSym NodeR) 0 closureTy)
\end{verbatim}
The first integer indicates the negative offset to the \texttt{Hp} pointer as in the last exercise. $\mathit{IndexSym}$ allows you to treat another symbol as an array to index into it. The second integer argument to $\mathit{IndexSym}$ indicates the offset into the array.
\item The $\mathit{withVar}$ function can be used to look up named symbols during code generation such as, for example, those generated by other functions like $\mathit{allocMemory}$. The first argument is the name of the symbol to look up and the second argument is a function which will be given the resolved symbol as an argument. The symbol value cannot leak out of this function.
\item The $\mathit{writeRegisterIx}$ function allows you to treat a register as an array.
\item $\mathit{forwardPtrTbl}$ is a $\mathit{Symbol}~\mathit{InfoTbl}$ which represents the info table for the forwarding pointer.
\item $\mathit{returnSymbol}$ may be used to generate a \texttt{return} statement in C.
\item If you ever need to express ``do nothing'', then $\mathit{return}~()$ might be for you.
\end{itemize} 

\part For each pointer in the free variables of a closure, the scavenging code calls the evacuation code for the pointer, then replaces it with the address of the new closure in to-space which is returned by the evacuation code, and finally calls the scavenging code. The corresponding function for this in the code generator is $\mathit{compScavenge}$ (at the bottom of \texttt{src/CodeGen.hs}). Complete the implementation of this function. Some hints:
\begin{itemize}
\item The scavenging code calls the evacuation and scavenging procedures of other closures, which expect the \texttt{Node} register to be set to their corresponding closures. This is a problem, because the scavenging code itself needs to remember which closure it belongs to. To circumvent this problem, you may want to back up the \texttt{Node} register into a local variable on entry, using \emph{e.g.} the $\mathit{loadLocalFromRegister}$ function. The $\mathit{Symbol}$ corresponding to the local variable is returned by the function.
\item The $\mathit{isPrimitive}$ function can be used to test if a type is primitive -- \emph{i.e.} not a pointer.
\item The $\mathit{callEvac}$ function will handle most of the work involved in calling another closure's evacuation code for you. It expects a symbol and an index as arguments. The symbol should be your backed-up version of \texttt{Node} and the index should be the index of the free variable which points to the other closure (1 for the first free variable, and so on).
\item The $\mathit{callScav}$ function will call the scavenging code of another closure. It expects the same arguments as the $\mathit{callEvac}$ function.
\end{itemize}
\end{parts}

\question The \texttt{run\_gc} procedure in \texttt{cbits/rts.c} will be invoked whenever the generated code detects that there is not enough space available on the heap. That is, when the heap overflow check fails. Modify \texttt{run\_gc} to contain your garbage collector. The rough algorithm for this as follows:
\begin{itemize}
\item If the active heap is \texttt{HeapA} (\texttt{Heap == HeapA}), switch to \texttt{HeapB} and vice-versa. You also need to update \texttt{Hp} to point at the start of the new heap and \texttt{HLimit} to point at the end.
\item Using the pointer stack as the root set, call the evacuation code followed by the scavenging code for each closure and update the entry in the pointer stack with the new address.
\item Call the evacuation code followed by the scavenging code for the closure pointed to by \texttt{Node} and update \texttt{Node} with the result of the evacuation.
\item Enter \texttt{Node} to continue execution of the program.
\end{itemize}
\end{questions}
Congratulations! If you have made it this far, you now have a relatively functional implementation of the STG machine that could be used as the target for a call-by-need functional language if you ever wanted to make one. 

\newpage
\appendix
\section{Appendix}

\subsection{Typed STG Language}

\begin{figure}[h]
\begin{mdframed}
    \begin{displaymath}
    \begin{array}{llcll}
    \multicolumn{5}{c}{\begin{array}{cc}
    \mathit{var} \in \texttt{[a-z][a-zA-Z0-9]*} & \mathit{ctr} \in \texttt{[A-Z][a-zA-Z0-9]*}
    \end{array}}\\\\
    \text{\bl{Program}} & \mathit{prog} & \bl{\to} & \mathit{typedecls}~\mathit{binds} \\
    \text{\bl{Data types}} & \mathit{typedecls} & \bl{\to} & \mathbf{type}~\mathit{ctr}_1~\mathit{vars}_1 = \mathit{ctrs}_1 \terminal{;} \ldots \terminal{;} \\
                           &                    &          & \mathbf{type}~\mathit{ctr}_n~\mathit{vars}_n = \mathit{ctrs}_n \terminal{;} & \bl{n \ge 0}\\
    \text{\bl{Constructors}} & \mathit{ctrs} & \bl{\to} & \mathit{ctr}_1~\mathit{types}_1 \terminal{|} \ldots \terminal{|} \mathit{ctr}_n~\mathit{types}_n & \bl{n \ge 1} \\
    \text{\bl{Type lists}} & \mathit{types} & \bl{\to} & \tau_1 \ldots \tau_n & \bl{n \ge 0}\\
    \text{\bl{Types}} & \tau, \sigma & \bl{\to} & \tau~\terminal{->}~\sigma \mid \tau~\sigma \mid (\tau) \mid \texttt{Int\#} \mid \mathit{var} \mid \mathit{ctr} & \\
    \text{\bl{Bindings}} & \mathit{binds} & \bl{\to} & \mathit{var}_1 \terminal{ = } \mathit{lf}_1\terminal{;} \ldots \terminal{;} \mathit{var}_n \terminal{ = } \mathit{lf}_n\terminal{;} & \bl{n \ge 1}\\
    \text{\bl{$\lambda$-forms}} & \mathit{lf} & \bl{\to} & \mathit{vars}_f~\terminal{\textbackslash} \pi~\mathit{vars}_a \terminal{ -> } \mathit{expr} \\
    \text{\bl{Update flag}} & \pi & \bl{\to} & \terminal{u} & \text{\bl{Updatable}} \\
    &     & \bl{\mid}     & \terminal{n} & \text{\bl{Not updatable}} \\
    \text{\bl{Expression}} & \mathit{expr} & \bl{\to}  & \terminal{let}~\mathit{binds}~\terminal{in}~\mathit{expr} & \text{\bl{Local definition}} \\
    &               & \bl{\mid} & \terminal{letrec}~\mathit{binds}~\terminal{in}~\mathit{expr} & \text{\bl{Local recursion}} \\
    &               & \bl{\mid} & \terminal{case}~\mathit{expr}~\terminal{of}~\mathit{alts} & \text{\bl{Case expression}} \\
    &               & \bl{\mid} & \mathit{var}~\mathit{atoms} & \text{\bl{Application}} \\
    &               & \bl{\mid} & \mathit{constr}~\mathit{atoms} & \text{\bl{Saturated constructor}} \\
    &               & \bl{\mid} & \mathit{prim}~\mathit{atoms} & \text{\bl{Saturated built-in operator}} \\
    &               & \bl{\mid} & \mathit{literal}  \\
    \text{\bl{Alternatives}} & \mathit{alts} & \bl{\to}  & \mathit{aalt}_1 \terminal{;} \ldots \terminal{;} \mathit{aalt}_n \terminal{;} \mathit{default} & \bl{n \ge 0} \text{ \bl{(Algebraic)}} \\
    &               & \bl{\mid} & \mathit{palt}_1 \terminal{;} \ldots \terminal{;} \mathit{palt}_n \terminal{;} \mathit{default} & \bl{n \ge 0} \text{ \bl{(Primitive)}} \\
    \text{\bl{Algebraic alt}} & \mathit{aalt} & \bl{\to}  & \mathit{constr}~\mathit{vars}~\terminal{->}~\mathit{expr} \\
    \text{\bl{Primitive alt}} & \mathit{palt} & \bl{\to}  & \mathit{literal}~\terminal{->}~\mathit{expr} \\
    \text{\bl{Default alt}} & \mathit{default} & \bl{\to}  & \mathit{var}~\terminal{->}~\mathit{expr} \\
    &                  & \bl{\mid} & \terminal{default ->}~\mathit{expr}\\
    \text{\bl{Literals}} & \mathit{literal} & \bl{\to}  & \terminal{0\#}~\bl{\mid}~\terminal{1\#}~\bl{\mid}~\ldots & \text{\bl{Primitive integers}}\\
    \text{\bl{Primitive ops}} & \mathit{prim} & \bl{\to}  & \terminal{+\#}~\bl{\mid}~\terminal{-\#}~\bl{\mid}~\terminal{*\#}~\bl{\mid}~\terminal{/\#}  & \text{\bl{Primitive integer ops}}\\     
    \text{\bl{Variable lists}} & \mathit{vars} & \bl{\to} & \terminal{\{}~\mathit{var}_1~\terminal{,} \ldots \terminal{,}~\mathit{var}_n~\terminal{\}} & \bl{n \ge 0}\\  
    \text{\bl{Atom lists}} & \mathit{atoms} & \bl{\to} & \terminal{\{}~\mathit{atom}_1~\terminal{,} \ldots \terminal{,}~\mathit{atom}_n~\terminal{\}} & \bl{n \ge 0}\\ 
    \text{\bl{Atoms}} & \mathit{atom}  & \bl{\to} & \mathit{var} ~\bl{\mid}~ \mathit{literal}            \\                        
    \end{array}
    \end{displaymath}
    \caption{Typed STG language}
    \label{fig:stglang}
\end{mdframed}
\end{figure} \pagebreak
\subsection{Operational Semantics}
\label{app:semantics}

Configurations of the Spineless Tagless G-Machine are six-tuples
\begin{displaymath}
\begin{array}{lcl}
\mathit{Config} & = & \langle \mathit{code}, \mathit{as}, \mathit{rs}, \mathit{us}, \mathit{h}, \sigma \rangle 
\end{array}
\end{displaymath}
where
\begin{itemize}
\item $\mathit{code}$ is the current ``instruction'' to be evaluated (see below)
\item $\mathit{as}$ is the argument stack, which contains values (see below)
\item $\mathit{rs}$ is the return stack, which contains continuations 
\item $\mathit{us}$ is the update stack, which contains update frames 
\item $h$ is the heap, which contains only closures 
\item $\sigma$ is the global environment, which maps the names of global definitions to memory addresses 
\end{itemize}

\subsubsection{Addresses}

Memory addresses $\mathit{a} \in \mathbb{N}$. Each location in memory can be occupied by exactly one closure.

\subsubsection{Global Environment}

The global environment $\sigma$ is a mapping of labels to memory addresses. In general, a program in the STG language is a collection of global definitions:
\begin{displaymath}
    \begin{array}{lcl}
    g_0 & = & \mathit{vs}_0~\terminal{\textbackslash}\pi~\mathit{xs}_0~\terminal{->}~\mathit{e}_0 \\
    \ldots \\
    g_n & = & \mathit{vs}_n~\terminal{\textbackslash}\pi~\mathit{xs}_n~\terminal{->}~\mathit{e}_n \\
    \end{array}
\end{displaymath} 
The corresponding global environment would be:
\begin{displaymath}
\begin{array}{lcl}
\sigma & = & \set{g_0 \mapsto a_0, \ldots, g_n \mapsto a_n}
\end{array}
\end{displaymath}
where $a_0 \ldots a_n$ are distinct memory addresses.

\subsubsection{Values}

\emph{Note: these aren't values in the operational semantics sense of the word, but rather different interpretations of integers in the operational semantics.}

We distinguish between address values (pointers) and primitive integer values:
\begin{center}
\begin{tabular}{p{1cm}p{6cm}}
$\mathit{Addr}~a$ & A memory address (pointer) \\
$\mathit{Int}~n$  & A primitive integer value
\end{tabular}
\end{center}

\subsubsection{Local Environments}

A local environment $\rho$ is a mapping of variable names to values. Such environments are created as the result of local bindings (\emph{e.g.} \texttt{let} and \texttt{letrec} bindings).

\subsubsection{Closures}

A closure is a structure on the heap, consisting of an arbitrary complex $\lambda$-form and a collection of free variables. There is one closure for every global definition and additional closures will be allocated as a program is evaluated.

\emph{Note: for the purpose of the interpreter, we never bother to remove closures from the heap and assume that there is an infinite amount of memory available.}

\subsubsection{Heap}

The heap $h$ is a mapping from memory address to closures. In general, a program in the STG language is a collection of global definitions:
\begin{displaymath}
    \begin{array}{lcl}
    g_0 & = & \mathit{vs}_0~\terminal{\textbackslash}\pi~\mathit{xs}_0~\terminal{->}~\mathit{e}_0 \\
    \ldots \\
    g_n & = & \mathit{vs}_n~\terminal{\textbackslash}\pi~\mathit{xs}_n~\terminal{->}~\mathit{e}_n \\
    \end{array}
\end{displaymath} 
The corresponding heap is:
\begin{displaymath}
\begin{array}{lcl}
h & = & \begin{array}[t]{llclll}
\{ & a_0 & \mapsto & (\mathit{vs}_0~\terminal{\textbackslash}\pi~\mathit{xs}_0~\terminal{->}~\mathit{e}_0) & \set{} & \\
   &     & \ldots  &  \\
   & a_n & \mapsto & (\mathit{vs}_n~\terminal{\textbackslash}\pi~\mathit{xs}_n~\terminal{->}~\mathit{e}_n) & \set{} & \}
\end{array}
\end{array}
\end{displaymath}
\emph{Note that global definitions shouldn't have free variables, so the collections of free variables for their closures on the heap are empty, indicated by the $\set{}$.}

\subsubsection{Code}

The code component of a configuration is one of the following states:
\begin{center}
\begin{tabular}{p{2.5cm}p{8cm}}
$\mathit{Eval}~e~\rho$             & Evaluate the expression $e$ in the local environment $\rho$ and apply its value to the arguments on the argument stack. The expression $e$ can be an arbitrarily complex expression in the STG language. \\
$\mathit{Enter}~a$                 & Apply the closure at address $a$ to the arguments on the argument stack. \\
$\mathit{ReturnCon}~c~\mathit{vs}$ & Return the constructor $c$ applied to the values $\mathit{vs}$ to the continuation on the return stack. \\
$\mathit{ReturnInt}~n$             & Return the primitive integer $n$ to the continuation on the return stack. 
\end{tabular}
\end{center}

\subsubsection{The Stacks}

There are three stacks in the STG machine.

\begin{itemize}
\item The argument stack 
\item The return stack
\item Whenever we encounter a lambda form whose update flag is set to \texttt{u}, \emph{i.e.} it can be updated, an \emph{update frame} is pushed onto the update stack. 
\end{itemize}

\subsubsection{Retriving the value of an atom}

Before we start defining the transition rules for the Spineless Tagless G-Machine, let us first define a little helper function which allows us to retrieve the values associated with atoms (see \autoref{fig:stglang}). If the atom is a primitive integer, then the corresponding STG value is also just a primitive integer. If the atom is a variable and it is an element of the domain of the local environment $\rho$, then we return the value associated with it. Otherwise, we try to look up its value in the global environment $\sigma$:
\begin{displaymath}
\begin{array}{lcll}
\mathit{val}~\rho~\sigma~\texttt{n} & = & \mathit{Int}~\texttt{n} & \\
\mathit{val}~\rho~\sigma~\texttt{x} & = & \rho~\texttt{x}   & \text{if}~\texttt{x} \in \mathit{dom}(\rho) \\
                                    &   & \sigma~\texttt{x} & \text{otherwise}
\end{array}
\end{displaymath}

\subsubsection{Application}

The first rule in the operational semantics concerns the application of some variable to some list of atoms. If we are in an $\mathit{Eval}$ state and the expression is of form $\mathit{f}~\mathit{xs}$ where $f$ is a variable, $\mathit{xs}$ is a list of atoms, and the value of $f$ is a memory address (see Rule \ref{eqn:varint} for the case where the value is an integer literal), then we push the values of the atoms in $\mathit{xs}$ onto the argument stack and move into an $\mathit{Enter}$ state for the closure represented by $f$:
\begin{mdframed}
\begin{equation}
\begin{array}{cl}
 & \langle \mathit{Eval}~(f~\mathit{xs})~\rho, \mathit{as}, \mathit{rs}, \mathit{us}, h, \sigma \rangle \\
 & \mathbf{where}~\mathit{val}~\rho~\sigma~f = \mathit{Addr}~a\\[0.25cm]
\Longrightarrow & \langle \mathit{Enter}~a, \overline{(\mathit{val}~\rho~\sigma~\mathit{xs})} \append \mathit{as}, \mathit{rs}, \mathit{us}, h, \sigma \rangle
\end{array}
\end{equation}
\end{mdframed}
The $\overline{\mathit{val}~\rho~\sigma~\mathit{xs}}$ notation is used to mean ``apply $\mathit{val}~\rho~\sigma$ to every element in the list $\mathit{xs}$''. The $\append$ operator is used to concatenate two lists.

The second rule concerns entering a closure. For now, we will only consider \emph{non-updatable} closures. The rule for updatable closures is given in Appendix \ref{app:updates}.
\begin{mdframed}
\begin{equation}
\begin{array}{cl}
 & \langle \mathit{Enter}~a, \mathit{as}, \mathit{rs}, \mathit{us}, h[a \mapsto \mathit{vs}~\terminal{\textbackslash}\pi~\mathit{xs}~\terminal{->}~\mathit{e}], \sigma \rangle \\
 & \mathbf{where}~\mathit{length}~\mathit{as} \geq \mathit{length}~\mathit{xs}\\[0.25cm]
\Longrightarrow & \langle \mathit{Eval}~e~\rho, \mathit{as}', \mathit{rs}, \mathit{us}, h, \sigma \rangle \\
 & \mathbf{where}~\begin{array}[t]{lcl}
  \mathit{ws}_a \append \mathit{as'} & = & \mathit{as} \qquad \textbf{such that}~\mathit{length}~\mathit{ws}_a = \mathit{length}~\mathit{xs} \\
  \rho & = & \hslist{\overline{\mathit{vs} \mapsto \mathit{ws}_f}, \overline{\mathit{xs} \mapsto \mathit{ws}_a}}
  \end{array}
\end{array}
\end{equation}
\end{mdframed}
When a non-updatable closure is entered, the local environment $\rho$ is constructed by binding the $\lambda$-form's free variables, $\mathit{vs}$, to the values, $\mathit{ws}_f$, found in the closure. The arguments, $\mathit{xs}$, are bound to the values found on top of the argument stack, $\mathit{ws}_a$. The resulting configuration indicates that the body of the closure should be evaluated next.

\subsubsection{\texttt{let(rec)} expressions}

A \texttt{let} expression constructs closures for all of its bindings on the heap:
\begin{mdframed}
\begin{equation}
\begin{array}{cl}
 & \langle \mathit{Eval}~\left( \begin{array}{llcl}
 \mathbf{let} & x_1 & = & \mathit{vs}_1~\terminal{\textbackslash}\pi_1~\mathit{xs}_1~\terminal{->}~\mathit{e}_1 \\
              & \multicolumn{3}{l}{\ldots} \\
              & x_n & = & \mathit{vs}_n~\terminal{\textbackslash}\pi_n~\mathit{xs}_n~\terminal{->}~\mathit{e}_n \\
 \multicolumn{4}{l}{\mathbf{in}~e}
 \end{array} \right)~\rho,\mathit{as},\mathit{rs},\mathit{us},h,\sigma \rangle \\[1.5cm]
\Longrightarrow & \langle \mathit{Eval}~e~\rho', \mathit{as},\mathit{rs},\mathit{us},h',\sigma \rangle \\
 & \mathbf{where}~\begin{array}[t]{lcl}
 \rho' & = & \rho[x_1 \mapsto \mathit{Addr}~a_1, \ldots, x_n \mapsto \mathit{Addr}~a_n] \\
 h' & = & h\left[\begin{array}{lcl}
 a_1 & \mapsto & (\mathit{vs}_1~\terminal{\textbackslash}\pi_1~\mathit{xs}_1~\terminal{->}~\mathit{e}_1)~(\rho_{\mathit{rhs}~\mathit{vs}_1}) \\
 \multicolumn{3}{l}{\ldots} \\
 a_n & \mapsto & (\mathit{vs}_n~\terminal{\textbackslash}\pi_n~\mathit{xs}_n~\terminal{->}~\mathit{e}_n)~(\rho_{\mathit{rhs}~\mathit{vs}_n})
 \end{array}\right] \\
 \rho_{\mathit{rhs}} & = & \rho
 \end{array}
\end{array}
\label{eqn:let}
\end{equation}
\end{mdframed}
First, we construct a new local environment, $\rho'$, by extending $\rho$ with mappings for all the binding names, $x_1 \ldots x_n$, to fresh addresses, $a_1 \ldots a_n$ -- \emph{i.e.} ones that have not been used for anything else yet. Then we construct a new heap, $h'$, by extending the old heap, $h$, with closures for all bindings. The free variables of the bindings are resolved using $\rho_{\mathit{rhs}}$, which is the same as the old local environment, $\rho$ -- \emph{i.e.} it does not yet contain mappings for the bindings in this \texttt{let} expression.

The rule for \texttt{letrec} is identical to Rule \ref{eqn:let}, except that $\rho_{\mathit{rhs}} = \rho'$ so that all bindings in the \texttt{letrec}-expression are in each others' scope:
\begin{mdframed}
\begin{equation}
\begin{array}{cl}
 & \langle \mathit{Eval}~\left( \begin{array}{llcl}
 \mathbf{letrec} & x_1 & = & \mathit{vs}_1~\terminal{\textbackslash}\pi_1~\mathit{xs}_1~\terminal{->}~\mathit{e}_1 \\
              & \multicolumn{3}{l}{\ldots} \\
              & x_n & = & \mathit{vs}_n~\terminal{\textbackslash}\pi_n~\mathit{xs}_n~\terminal{->}~\mathit{e}_n \\
 \multicolumn{4}{l}{\mathbf{in}~e}
 \end{array} \right)~\rho,\mathit{as},\mathit{rs},\mathit{us},h,\sigma \rangle \\[0.25cm]
\Longrightarrow & \langle \mathit{Eval}~e~\rho', \mathit{as},\mathit{rs},\mathit{us},h',\sigma \rangle \\
 & \mathbf{where}~\begin{array}[t]{lcl}
 \rho' & = & \rho[x_1 \mapsto \mathit{Addr}~a_1, \ldots, x_n \mapsto \mathit{Addr}~a_n] \\
 h' & = & h\left[\begin{array}{lcl}
 a_1 & \mapsto & (\mathit{vs}_1~\terminal{\textbackslash}\pi_1~\mathit{xs}_1~\terminal{->}~\mathit{e}_1)~(\rho_{\mathit{rhs}~\mathit{vs}_1}) \\
 \multicolumn{3}{l}{\ldots} \\
 a_n & \mapsto & (\mathit{vs}_n~\terminal{\textbackslash}\pi_n~\mathit{xs}_n~\terminal{->}~\mathit{e}_n)~(\rho_{\mathit{rhs}~\mathit{vs}_n})
 \end{array}\right] \\
 \rho_{\mathit{rhs}} & = & \rho'
 \end{array}
\end{array}
\label{eqn:let}
\end{equation}
\end{mdframed}

\subsubsection{Case expressions and data constructors}

The return stack is used for the first time when we come to \texttt{case} expressions. Given the expression
\begin{displaymath}
\texttt{case}~e~\texttt{of}~\mathit{alts}
\end{displaymath}
the operational interpretation is ``push a continuation onto the return stack, and evaluate $e$''. When the evaluation of $e$ is complete, execution will resume at the continuation, which then decides which alternative to execute. The rule for \texttt{case} follows fairly directly:
\begin{mdframed}
\begin{equation}
\begin{array}{cl}
 & \langle \mathit{Eval}~(\texttt{case}~e~\texttt{of}~\mathit{alts})~\rho, \mathit{as}, \mathit{rs}, \mathit{us}, h, \sigma \rangle \\[0.25cm]
\Longrightarrow & \langle \mathit{Eval}~e~\rho, \mathit{as}, (\mathit{alts}, \rho) : \mathit{rs}, \mathit{us}, h, \sigma \rangle 
\end{array}
\end{equation}
\end{mdframed}
The continuation is a pair $(\mathit{alts}, \rho)$; the alternatives $\mathit{alts}$ say what to do when evaluation of $e$ completes, while the local environment $\rho$ provides the context in which to evaluate the chosen alternative. 

The other side of the coin is the rules for constructors and literals. Presumably, $e$ eventually evaluates to either a constructor or a literal, at which point the continuation must be popped from the return stack and executed. The rules for constructors and literals each use an intermediate state, $\mathit{ReturnCon}$ and $\mathit{ReturnInt}$ respectively, just as the rule for function application uses $\mathit{Enter}$. Primitive values are dealt with in the next section, while the rules for constructors are given next.

Evaluating a constructor application simply moves into the $\mathit{ReturnCon}$ state:
\begin{mdframed}
\begin{equation}
\begin{array}{cl}
 & \langle \mathit{Eval}~(c~\mathit{xs})~\rho, \mathit{as}, \mathit{rs}, \mathit{us}, h, \sigma \rangle \\[0.25cm]
\Longrightarrow & \langle \mathit{ReturnCon}~c~(\overline{\mathit{val}~\rho~\sigma~\mathit{xs}}), \mathit{as}, \mathit{rs}, \mathit{us}, h, \sigma \rangle 
\end{array}
\end{equation}
\end{mdframed}
The rules for $\mathit{ReturnCon}$ return to the appropriate continuation, taken from the return stack:
\begin{mdframed}
\begin{equation}
\begin{array}{cl}
 & \langle \mathit{ReturnCon}~c~\mathit{ws}, \mathit{as},  (\ldots \texttt{;} c~\mathit{vs}~\texttt{->}~e \texttt{;} \ldots, \rho) : \mathit{rs}, \mathit{us}, h, \sigma \rangle \\[0.25cm]
\Longrightarrow & \langle \mathit{Eval}~e~\rho[\overline{\mathit{vs} \mapsto \mathit{ws}}], \mathit{as}, \mathit{rs}, \mathit{us}, h, \sigma \rangle
\end{array}
\label{eq:ctrmatch}
\end{equation}
\end{mdframed}
Provided that the continuation on the return stack contains a pattern $c~\mathit{vs}$ whose constructor $c$ is the same as that being evaluated, we just evaluate the right-hand side of that alternative, $e$, in the saved local environment, $\rho$, augmented with bindings for the pattern variables, $\mathit{vs}$, to the values of the actual arguments to $c$, represented by $\mathit{ws}$.

If there is no such alternative in the continuation on top of the return stack, then the default alternative is taken. If the default alternative does not bind a variable, then the rule is easy:
\begin{mdframed}
\begin{equation}
\begin{array}{cl}
 & \langle \mathit{ReturnCon}~c~\mathit{ws}, \mathit{as}, \left(\begin{array}{lcl}
 c_1~\mathit{vs}_1 & \texttt{->} & e_1 \texttt{;} \\
 \multicolumn{3}{l}{\ldots \texttt{;}} \\
 c_n~\mathit{vs}_n & \texttt{->} & e_n \texttt{;} \\
 \texttt{default} & \texttt{->} & e_d
 \end{array}, \rho\right) : \mathit{rs}, \mathit{us}, h, \sigma \rangle \\
 & \mathbf{where}~\forall i \in \mathbb{N}~\text{such that}~1 \leq i \leq n, c \neq c_i \\[0.25cm]
\Longrightarrow & \langle \mathit{Eval}~e_d~\rho, \mathit{as}, \mathit{rs}, \mathit{us}, h, \sigma \rangle
\end{array}
\end{equation}
\end{mdframed}
We simply move to a state where the expression on the right-hand side of the default alternative, $e_d$, is the next expression to evaluate.

However, if the default alternative binds a variable $v$, we need to allocate a constructor closure on the heap. We do this by extending the local environment with a mapping for $v$ to the address of a new closure, containing a $\lambda$-form whose free variables are the constructor's arguments and whose body is the constructor $c$ applied to those variables. That closure is then added to the old heap, $h$, to form a new heap, $h'$: 
\begin{mdframed}
\begin{equation}
\begin{array}{cl}
 & \langle \mathit{ReturnCon}~c~\mathit{ws}, \mathit{as}, \left(\begin{array}{lcl}
 c_1~\mathit{vs}_1 & \texttt{->} & e_1 \texttt{;} \\
 \multicolumn{3}{l}{\ldots \texttt{;}} \\
 c_n~\mathit{vs}_n & \texttt{->} & e_n \texttt{;} \\
 v & \texttt{->} & e_d
 \end{array}, \rho\right) : \mathit{rs}, \mathit{us}, h, \sigma \rangle \\
 & \mathbf{where}~\forall i \in \mathbb{N}~\text{such that}~1 \leq i \leq n, c \neq c_i \\[0.25cm]
\Longrightarrow & \langle \mathit{Eval}~e_d~\rho[v \mapsto a], \mathit{as}, \mathit{rs}, \mathit{us}, h', \sigma \rangle\\
 & \mathbf{where}~\begin{array}[t]{l}
 h' = h[a \mapsto (\mathit{vs}~\terminal{\textbackslash}\texttt{n}~\texttt{\set{}}~\terminal{->}~\mathit{c}~\mathit{vs})~\mathit{ws}] \\
 \mathit{vs}~\text{is a sequence of arbitrary, distinct variables} \\
 \mathit{length}~\mathit{vs} = \mathit{length}~\mathit{ws}
 \end{array}
\end{array}
\end{equation}
\end{mdframed}

\subsubsection{Built-in operations}

In this section we give the extra rules which handle primitive values. The rule for evaluating a primitive literal, $k$, enters the $\mathit{ReturnInt}$ state:
\begin{mdframed}
\begin{equation}
\begin{array}{cl}
 & \langle \mathit{Eval}~k~\rho, \mathit{as}, \mathit{rs}, \mathit{us}, h, \sigma \rangle \\[0.25cm]
\Longrightarrow & \langle \mathit{ReturnInt}~k, \mathit{as}, \mathit{rs}, \mathit{us}, h, \sigma \rangle 
\end{array}
\end{equation}
\end{mdframed}
A similar rule deals with the case where a variable bound to a primitive value is entered:
\begin{mdframed}
\begin{equation}
\begin{array}{cl}
 & \langle \mathit{Eval}~(f~\texttt{\set{}})~\rho[f \mapsto \mathit{Int}~k], \mathit{as}, \mathit{rs}, \mathit{us}, h, \sigma \rangle \\[0.25cm]
\Longrightarrow & \langle \mathit{ReturnInt}~k, \mathit{as}, \mathit{rs}, \mathit{us}, h, \sigma \rangle 
\end{array}
\label{eqn:varint}
\end{equation}
\end{mdframed}
Next come the rules for the $\mathit{ReturnInt}$ state, which look for a continuation on the return stack. First, the case where there is an alternative which matches the literal (this corresponds to Rule \ref{eq:ctrmatch} for constructors):
\begin{mdframed}
\begin{equation}
\begin{array}{cl}
 & \langle \mathit{ReturnInt}~k, \mathit{as}, (\ldots \texttt{;} k~\texttt{->}~e \texttt{;} \ldots, \rho) : \mathit{rs}, \mathit{us}, h, \sigma \rangle \\[0.25cm]
\Longrightarrow & \langle \mathit{Eval}~e~\rho, \mathit{as}, \mathit{rs}, \mathit{us}, h, \sigma \rangle 
\end{array}
\end{equation}
\end{mdframed}
Next, the cases where the default alternative is taken:
\begin{mdframed}
\begin{equation}
\begin{array}{cl}
 & \langle \mathit{ReturnInt}~k, \mathit{as}, \left(\begin{array}{lcl}
 k_1 & \texttt{->} & e_1 \texttt{;} \\
 \multicolumn{3}{l}{\ldots \texttt{;}} \\
 k_n & \texttt{->} & e_n \texttt{;} \\
 x & \texttt{->} & e_d
 \end{array}, \rho\right) : \mathit{rs}, \mathit{us}, h, \sigma \rangle \\
 & \mathbf{where}~\forall i \in \mathbb{N}~\text{such that}~1 \leq i \leq n, k \neq k_i \\[0.25cm]
\Longrightarrow & \langle \mathit{Eval}~e_d~\rho[x \mapsto \mathit{Int}~k], \mathit{as}, \mathit{rs}, \mathit{us}, h, \sigma \rangle
\end{array}
\end{equation}
\end{mdframed}
\begin{mdframed}
\begin{equation}
\begin{array}{cl}
 & \langle \mathit{ReturnInt}~k, \mathit{as}, \left(\begin{array}{lcl}
 k_1 & \texttt{->} & e_1 \texttt{;} \\
 \multicolumn{3}{l}{\ldots \texttt{;}} \\
 k_n & \texttt{->} & e_n \texttt{;} \\
 \texttt{default} & \texttt{->} & e_d
 \end{array}, \rho\right) : \mathit{rs}, \mathit{us}, h, \sigma \rangle \\
 & \mathbf{where}~\forall i \in \mathbb{N}~\text{such that}~1 \leq i \leq n, k \neq k_i \\[0.25cm]
\Longrightarrow & \langle \mathit{Eval}~e_d~\rho, \mathit{as}, \mathit{rs}, \mathit{us}, h, \sigma \rangle
\end{array}
\end{equation}
\end{mdframed}
Finally, we need a family of rules for built-in arithmetic operations which, for each binary built-in operation $\oplus$ have the form:
\begin{mdframed}
\begin{equation}
\begin{array}{cl}
 & \langle \mathit{Eval}~(\oplus~\texttt{\set{$x_1\texttt{,} x_2$}})~\rho[x_1 \mapsto \mathit{Int}~i_1, x_2 \mapsto \mathit{Int}~i_2], \mathit{as}, \mathit{rs}, \mathit{us}, h, \sigma \rangle \\[0.25cm]
\Longrightarrow & \langle \mathit{ReturnInt}~(i_1 \oplus i_2)~\rho, \mathit{as}, \mathit{rs}, \mathit{us}, h, \sigma \rangle 
\end{array}
\end{equation}
\end{mdframed}
The $\oplus$ symbol is used to mean one of \texttt{+\#}, \texttt{-\#}, \texttt{*\#}, or \texttt{/\#} as well as the corresponding arithmetic operators, respectively.

\subsubsection{Updates}
\label{app:updates}

Updates happen in two stages:
\begin{enumerate}
\item When an updatable closure is entered, it pushes an \emph{update frame} onto the update stack, and makes the argument and return stacks empty. An update frame is a triple $(\mathit{as}_u, \mathit{rs}_u, a_u)$, consisting of:
\begin{itemize}
\item $\mathit{as}_u$, the previous argument stack;
\item $\mathit{rs}_u$, the previous return stack;
\item $a_u$, the pointer to the closure being entered, and which should later be updated.
\end{itemize}
\item When evaluation of the closure is complete, an update is triggered. This can happen in one of two ways:
\begin{itemize}
\item If the value of the closure is a data constructor or literal, an attempt will be made to pop a continuation from the return stack, which will fail because the return stack is empty. This failure triggers an update.
\item If the value of the closure is a function, the function will attempt to bind arguments which are not present on the argument stack, because they were moved into the update frame. This failure to find enough arguments triggers an update.
\end{itemize}
\end{enumerate}
If we attempt to enter a closure whose update flag is set to \texttt{u} to indicate that it can be updated, we push an update frame onto the update stack:
\begin{mdframed}
\begin{equation}
\begin{array}{cl}
 & \langle \mathit{Enter}~a, \mathit{as}, \mathit{rs}, \mathit{us}, h[a \mapsto (\mathit{vs}~\terminal{\textbackslash}\terminal{u}~\set{}~\terminal{->}~\mathit{e})~\mathit{ws}], \sigma \rangle \\[0.25cm]
\Longrightarrow & \langle \mathit{Eval}~e~\rho, \set{}, \set{}, (\mathit{as},\mathit{rs},a) : \mathit{us}, h, \sigma \rangle \\
 & \mathbf{where}~\begin{array}[t]{lcl}
 \rho & = & \hslist{\mathit{vs}_0 \mapsto \mathit{ws}_0, \ldots, \mathit{vs}_n \mapsto \mathit{ws}_n}
 \end{array}
\end{array}
\end{equation}
\end{mdframed}
Next, we need rules for constructors which see an empty return stack. When this happens, they update the closure pointed to by the update frame, restore the argument and return stacks from the update frame, and try again. It may be that the restored return stack contains the continuation, but it too may be empty, in which case a second update is performed, and so on until the continuation is exposed:
\begin{mdframed}
\begin{equation}
\begin{array}{cl}
 & \langle \mathit{ReturnCon}~c~\mathit{ws}, \set{}, \set{}, (\mathit{as}_u, \mathit{rs}_u, a_u) : \mathit{us}, h, \sigma \rangle \\[0.25cm]
\Longrightarrow & \langle \mathit{ReturnCon}~c~\mathit{ws}, \mathit{as}_u, \mathit{rs}_u, \mathit{us}, h_u, \sigma \rangle \\
 & \mathbf{where}~\begin{array}[t]{lcl}
 \multicolumn{3}{l}{\mathit{vs}~\text{is a sequence of arbitrary, distinct variables}} \\
 \multicolumn{3}{l}{\mathit{length}~\mathit{vs} = \mathit{length}~\mathit{ws}} \\
 h_u & = & h[a_u \mapsto (\mathit{vs}~\terminal{\textbackslash}\texttt{n}~\set{}~\terminal{->}~\mathit{c}~\mathit{vs})~\mathit{ws}]
 \end{array}
\end{array}
\end{equation}
\end{mdframed}
The closured to be updated at address $a_u$ is just updated with a standard constructor closure. Only a rule for $\mathit{ReturnCon}$ needs to be given. It is not possible for the $\mathit{ReturnInt}$ state to see an empty return stack, because that would imply that a closure should be updated with a primitive value, but no closure has a primitive type.

Finally, we need a rule to handle the case where there are not enough arguments on the stack to be bound by a $\lambda$-abstraction ($\mathit{length}~\mathit{as} < \mathit{length}~\mathit{xs}$), which triggers an update:
%Note that the pre-condition for this rule is the same as for Rule \ref{}, but with $\mathit{length}~\mathit{as} < \mathit{length}~\mathit{xs}$: 
\begin{mdframed}
\begin{equation}
\begin{array}{cl}
 & \langle \mathit{Enter}~a, \mathit{as}, \set{}, (\mathit{as}_u, \mathit{rs}_u, a_u) : \mathit{us}, h[a \mapsto \mathit{vs}~\terminal{\textbackslash}\terminal{n}~\mathit{xs}~\terminal{->}~\mathit{e}], \sigma \rangle \\
 & \mathbf{where}~\mathit{length}~\mathit{as} < \mathit{length}~\mathit{xs}\\[0.25cm]
\Longrightarrow & \langle \mathit{Enter}~a, \mathit{as} \append \mathit{as}_u, \mathit{rs}_u, \mathit{us}, h_u, \sigma \rangle \\
 & \mathbf{where}~\begin{array}[t]{lcl}
 \mathit{xs}_1 \append \mathit{xs}_2 & = & xs \qquad \textbf{such that}~\mathit{length}~\mathit{xs}_1 = \mathit{length}~\mathit{as} \\
 h_u & = & h[(a_u \mapsto (\mathit{vs} \append \mathit{xs}_1)~\terminal{\textbackslash}\terminal{n}~\mathit{xs}_2~\terminal{->}~e)~(\mathit{ws}_f \append \mathit{as})]
 \end{array}
\end{array}
\end{equation}
\end{mdframed}

\subsection{Runtime}
\label{app:runtime}

\subsubsection{Heap}
\label{app:heap}

We are making use of C's heap to store static closures and info tables. For dynamic closures, we use our own heap. The heap pointer, \texttt{Hp}, points at the first free location in our heap. The heap is simply an array and grows from lower to higher memory addresses.

\subsubsection{Stacks}
\label{app:stacks}

The STG-machine's three stacks can be merged into a single stack at runtime. However, for reasons which will become apparent in the final supervision, we split that resulting stack into two new stacks: a value stack and a pointer stack. The pointer stack contains only pointers to closures and the value stack contains primitive values, pointers to continuations, and so on. To save memory at runtime, both stacks will occupy the same region in memory. The initial memory layout of the stacks is shown in \autoref{fig:stacklayout}. The pointer stack grows from lower to higher memory addresses and the value stack grows from lower to higher memory addresses. 

Initially, the pointer stack is completely empty. The pointer stack pointer, \texttt{SpPtr}, points to the memory address at which the stacks are located. The value stack contains a single item: a pointer to a continuation, \texttt{finished} (defined in \texttt{cbits/rts.c}), which, when invoked, terminates the program. The value stack pointer, \texttt{SpVal}, points to the next available slot. Both stack pointers should point to the next available slot after a push or pop. We can detect a stack overflow by checking whether \texttt{SpPtr} is greater than \texttt{SpVal} or vice-versa.

\begin{figure}
\begin{center}
\begin{drawstack}
  % Within the environment, draw stack elements with \cell{...}
  \cell{\texttt{}} \cellptr{\texttt{SpPtr} and \texttt{Stack}}
  \padding{3}{$\vdots$}
  \cell{} \cellptr{\texttt{SpVal}}
  \cell{\texttt{finished}} 
\end{drawstack}
\caption{Initial stack layout}
\label{fig:stacklayout}
\end{center}
\end{figure}


\subsubsection{Registers}
\label{app:registers}

Aside from the heap pointer, \texttt{Hp}, and the two stack pointers, \texttt{SpPtr} and \texttt{SpVal}, there are three other registers: the \texttt{Ret} register stores the primitive integer of the last $\mathit{ReturnInt}$ state, the \texttt{RetVec} register stores the pointer to the last return vector, and the \texttt{Node} register stores the memory address of the closure that was last entered. All registers are implemented as global variables in C (in \texttt{cbits/rts.h}).

\end{document}
